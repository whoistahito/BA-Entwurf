% ---------------------------------------------------------------------
% Das Dokument kompiliert mit pdflatex und ist auf Basis 
% von Koma-Script entstanden. 
%
% Autor des Templates (für Anmerkungen): 
% Michael von Riegen, riegen@informatik.uni-hamburg.de
%
% Einzelne Code-Teile für das Titelblatt sind aus dem Template 
% von Benjamin Kirchheim entnommen.
%
% 25.05.09, Frank Langanke: Vorlage auf aktuelle KOMA-Version aktualisiert
% 26.05.09, Michael von Riegen: Anmerkung --> aktuelles Koma-Script ist nötig!
% 17.10.2016 neues Uni logo
% ---------------------------------------------------------------------
\documentclass[11pt,DIV=15,BCOR=20mm,bibliography=totoc]{scrbook}

% Import von Paketen und Optionen die das gesamte Dokument betreffen
% sind in myPreamble.sty ausgelagert.
\usepackage{myPreamble}
   
% Arbeitet man nur an einem Kapitel, wird durch folgenden Befehl nur dieses eingebunden.
% Spart manuelles auskommentieren von vielen include-Befehlen;
% hat keine Auswirkung auf input-Befehle
% \includeonly{introduction}
   
\begin{document}

% TITELSEITE
\begin{titlepage}

	% Fehler "destination with the same identifier" unterdrücken...
  \setcounter{page}{-1}

	% Titelseite
	\begin{figure}[h]
		\begin{minipage}[b]{62mm}
			\includegraphics[width=62mm]{images/unilogo.png}
		\end{minipage}
		\hspace{4cm}
		%\begin{minipage}[b]{59mm}
		%	\includegraphics[width=59mm]{images/minlogo}
		%\end{minipage}
	\end{figure}

	\vfill
	
	\begin{center}
		% Diplomarbeit 
		\noindent { \huge
			Bachelor Thesis \\
		}
		\vspace{14mm}
		% Titel
		\noindent \textbf{\huge
            Novel Approach for a Job Matching System: Large Language Model-Based Requirements Extraction and Token-Embedding-Driven Similarity Search}
		\vspace{60mm}	
	\end{center}
	
	\vfill
	
	\noindent \textbf{Seyed Taha Amirhosseini} \\
	\noindent \rule{\textwidth}{0.4mm} 
	\noindent{\textrm{seyed.taha.amirhosseini@studium.uni-hamburg.de}} \\
	\noindent{\textrm{Course of Studies: Information Systems}} \\
	\noindent{\textrm{Matriculation Number: 7364370}} \\
	\begin{tabbing}
	\hspace{8em} \=  \kill
	Primary Referee:  Dr. Lothar Hotz \\
	Secondary Referee: Marten Borchers  \\
	~ \\
	Submission Date: 
	\end{tabbing}
	
	% Rückseite der Titelseite mit Zitat
	\newpage 
	\thispagestyle{empty}
	\setcounter{page}{0}

	% wenn man Lust auf ein Zitat hat...
	% ... ansonsten auskommentieren
	% ~\\ \vfill \noindent 
	% A distributed system is one where the failure of some \\
	% computer I've never heard of can keep me from getting my work done. \\
	% \textit{-- Leslie Lamport}
\end{titlepage}



% VERZEICHNISSE (Inhaltsverzeichnis, Abkürzungen)
% Vorspann einleiten --> Seitennummerierung römisch
\frontmatter

% Inhaltsverzeichnis
\tableofcontents
\cleardoublepage

% Hauptteil einleiten --> Seitennummerierung wieder arabisch
\mainmatter

\chapter{Introduction}
In the digital age, the proliferation of online job platforms has created a paradox: while job seekers have access to an unprecedented volume of opportunities, finding truly suitable positions remains a significant challenge \cite{challenge_recruitment}. One of the issues stems from the  fundamental search and filtering mechanisms employed by many platforms \cite{challenge_recruitment}. The predominant use of job titles as a search filter frequently leads to a mismatch between the search query and the functional requirements of the resulting job listings, which occurs because titles alone cannot capture the semantic similarity between a user's query and the actual tasks described in a role \cite{kabir2025ordinal}, presenting users with  job postings that are nominally correct but a poor match for their actual skills and qualifications \cite{womark2025lessons}. Another key problem is a phenomenon known as bid dominance, where less relevant jobs are ranked higher due to their higher submitted bids \cite{tradingrelevancerevenuejobs}.

\section{Motivation}
As technology evolves, jobs and their corresponding requirements become increasingly specific. Musazade et al. highlight this trend, noting that ``the study suggests that data professionals have started to be required to possess a narrower competence and specialization in particular common, unique and extensive tools and technologies'' \cite{Musazade2023toolsAndTech}. As pointed out previously, this increasing specialization creates challenges for online job platforms, which often struggle with matching jobs to users who fit their profiles. For example, Kabir et al. mention in their work that ``two queries such as `software engineer Java' and `Java backend developer' might have different token representations but are semantically similar due to overlapping skill requirements'' \cite{kabir2025ordinal}. Another example is from Womark et al., who note that ``the query `Human Resources' often returned unrelated jobs matching the term only in descriptions, in clauses such as `speak to a Human Resources Rep'.'' \cite{womark2025lessons}.\\
To overcome the limitations of title-based search, Kabir et al. suggest that ``contextual information, which provides a richer context to compare two job queries, such as skills associated with a job, or the specific industry sector to which the job belongs to, need to be incorporated in the similarity prediction model'' \cite{kabir2025ordinal}. Based on this, the approach of this thesis is to analyze the full text of job descriptions and match them based on a similarity search. \\

This research builds upon YourJobFinder, a pre-existing job aggregation tool, for the initial data acquisition phase. YourJobFinder scrapes unstructured job descriptions from various online sources based on a job title. However, this initial collection, which relies solely on title-based matching, suffers from low precision and includes many irrelevant postings. The core technical contribution of this thesis is a novel filtering and matching system that addresses this limitation. This system takes the raw, unstructured text provided by YourJobFinder as input and applies an intelligent filtering to identify job postings that match a user's profile.\\

Therefore, the objective of this thesis is to design and implement a job matching system that uses a Large Language Model (LLM) to extract key attributes—such as skills, experience, and qualifications—from job descriptions. This information is then converted into a vector representation. The system performs a comparison between the job's vector and a similarly vectorized user profile. This thesis aims to design an efficient system capable of producing accurate and reliable results using open-source models and tools, ensuring accessibility and cost-effectiveness.


\section{Questions}
To guide this research and systematically address the stated objective, this thesis poses the following questions:

\begin{itemize}
\item RQ1: How effectively can open-source Large Language Models extract structured job requirements (skills, experiences, qualifications) from unstructured job descriptions?


\item RQ2: Which open-source LLM demonstrates the best performance in terms of correctness, completeness, and alignment for the task of requirements extraction?


\end{itemize}


\chapter{State of the Art}

The limitations of current job-matching systems, as outlined in the introduction, have led researchers to explore more sophisticated approaches capable of capturing the semantic relationships between job descriptions, requirements, and user profiles. Over the past years, the field has evolved from simple keyword-based methods toward advanced semantic, transformer-based, and large-scale industrial systems. This chapter provides an overview of these developments, focusing on those techniques that inform the design of the system proposed in this thesis.


\section{Semantic and Transformer-Based Job Matching}

Early improvements over keyword matching emerged through semantic similarity models, which aimed to interpret the conceptual relevance between resumes and job descriptions. Research by Ajjam and Al-Raweshidy \cite{ajjam_ai-driven_2025} demonstrated that these approaches outperform keyword-based methods, achieving significantly higher similarity scores and enabling a more nuanced interpretation of the conceptual relevance between resumes and job descriptions.

The adoption of transformer models marked a major shift in this domain. Architectures such as BERT and its derivatives \cite{sentence-bert} enabled richer contextual understanding and improved generalization across career‑related tasks. The development of domain-specific models represented a pivotal advancement in this field. For instance, Rosenberger et al. proposed CareerBERT, a novel approach that leverages the power of unstructured textual data sources, that uses jobGBERT \cite{gbert} to embed resumes and job categories derived from the standardized European Skills, Competences, and Occupations (ESCO) taxonomy \cite{careerbert}. By pre-training on specialized career data, such models create a shared embedding space that more accurately captures the intricacies of professional qualifications. Their approach demonstrated comparable, though more reliable, performance to that of GPT-4 while exhibiting significantly superior computational efficiency \cite{careerbert}.

These developments highlight the steady shift toward embedding-based architectures capable of encoding job content beyond surface-level keywords. However, most domain-specific transformer models rely heavily on curated data sources or custom pretraining pipelines, which limits accessibility.


\section{Industrial-Scale Recommender Architectures}

Parallel to academic research, large technology companies have integrated semantic embeddings into production‑level recommender systems. CareerBuilder, for example, developed an embedding-based recommender system that constructs fused embeddings from text, semantic entities, and location data to handle matching at a massive scale \cite{zhao2021embeddingbasedrecommender}. Their two-stage process, involving approximate nearest neighbor search followed by a reranking model, led to significant gains in user engagement and match quality.
LinkedIn's LinkSAGE framework represents a significant advancement in the field of network analysis, incorporating Graph Neural Networks (GNN) to model the intricate and interwoven relationships within its professional network \cite{linksage}. LinkSAGE is using GraphSAGE architecture, which is  ``a general inductive framework that leverages node feature information (e.g., text attributes) to efficiently generate node embeddings for previously unseen data'' \cite{graphsage}. By conceptualizing members, employment opportunities, and competencies as nodes in a comprehensive graph, LinkSAGE is able to capture implicit signals and deliver highly relevant recommendations. Despite its effectiveness, this architecture poses significant barriers for broader adoption: it depends on massive proprietary datasets and requires extensive computational infrastructure, making it unsuitable for open, platform-aggregating systems such as the one developed in this thesis.
Nevertheless, LinkSAGE demonstrates a key insight for the field: effective job matching requires modeling the detailed relationships between applicant skills and job requirements. This principle directly informs the methodology of the approach proposed in this thesis.


\section{Addressing Data Challenges: Augmentation and Hard-Negative Mining}

As semantic models grew more advanced, researchers increasingly focused on addressing data sparsity and imbalance—persistent challenges in recruitment datasets. The CONFIT V2 framework \cite{confitv2} introduces a novel strategy to enrich training data by using LLMs to generate hypothetical reference resumes. These synthetic examples are combined with Runner-Up Hard-Negative Mining, a technique that selects job–resume pairs which appear similar but are incorrect, forcing the model to learn subtle distinctions between suitable and unsuitable matches. This method significantly improves model robustness and highlights the growing role of generative AI in enhancing contrastive learning processes.

Although such methods push performance forward, they require substantial computational resources and specialized training pipelines. This dependency limits their applicability in lightweight, low-budget, or open-source contexts.


\section{Related Work}
Following the broader developments in job matching, this section examines research directly related to the two core techniques used in this thesis: (1) LLM‑based requirements extraction and (2) embedding-driven similarity search.

\subsubsection{Requirements Extraction using LLMs}
LLMs have demonstrated strong capabilities in transforming unstructured job descriptions into structured information. Howison et al. \cite{howison2024} demonstrated the use of generative AI to extract structured labor market data, such as education requirements and job types, from real-world job ads with statistically reliable results. Similarly, Herandi et al. \cite{herandiskilllm} proposed Skill-LLM, a fine-tuned LLM optimized for skill extraction, significantly outperforming standard Named Entity Recognition (NER) baselines in identifying hard skills.
While fine-tuned models achieve strong performance, they require task-specific annotations and training. In contrast, methods such as those explored by Nguyen et al. demonstrate that general-purpose LLMs can achieve competitive results through few-shot prompting, particularly when dealing with syntactically complex skill mentions \cite{nguyen2024rethinkingskillextractionjob}. This reinforces the choice of this thesis to rely on general-purpose, open-source LLMs without any fine-tuning, thereby increasing accessibility and adaptability.

\subsection{Embedding and Similarity Search in Job Matching}
\label{introduction:similaritysearch}
Embedding-based similarity search has become a cornerstone of modern job recommendation systems. Domain-specific models such as CareerBERT \cite{careerbert} achieve strong performance by leveraging structured taxonomies. However, general-purpose models such as sentence-transformers (e.g., all-MiniLM-L6-v2) have also shown excellent retrieval performance without task‑specific training, as demonstrated by Kurek et al. \cite{kurek2024}. Building on this foundation, this thesis employs the model to process extracted requirements, enabling the system to capture semantic relationships effectively.

\section{Research Gap} \label{sota:research_gap}


The literature reviewed in this chapter demonstrates considerable
progress in semantic job matching, domain‑specific transformer models,
graph‑based recommender architectures, and data‑augmentation techniques.
 Despite these advancements, several limitations persist:

\begin{itemize}
    \item Domain‑specific models frequently depend on extensive
preprocessing pipelines or proprietary datasets, which restricts their
accessibility and reproducibility.
    \item Industrial‑scale recommender systems require significant
computational infrastructure and rely on large, platform‑specific
datasets, making them unsuitable for deployment by smaller organizations
 or independent aggregators.
    \item Most of the state-of-the-art approaches rely on end-to-end embedding models that function as ``black boxes.'' While effective, these systems often obscure the specific rationale behind a match (e.g., whether a score is driven by skills, experience, or qualification), making it difficult to verify or explain recommendations.

\end{itemize}

To make these differences more explicit,
\autoref{tab:model_comparison} contrasts commercial job portals,
representative academic approaches, and the method proposed in this
thesis. The comparison highlights how this work distinguishes itself
through a combination of semantic capabilities, open‑source components,
and the absence of fine‑tuning requirements.


\begin{table}[h]
    \caption{Systematic Comparison of Job Matching Approaches}
    \label{tab:model_comparison}
    \centering
    \begin{tabular*}{\textwidth}{| l@{\extracolsep\fill} r || c | c | c | c |} % Added extra column
        \hline

        Tools &
        \rotatebox{90}{Characteristics} &
        \rotatebox{90}{Uses LLMs} &
        \rotatebox{90}{Open-Source} &
        \rotatebox{90}{Training Required} &
        \rotatebox{90}{Interpretability} \\

        \hline
        \hline

        \multicolumn{2}{|l||}{LinkSage} &
        \xmark  &
        \xmark  &
        \cmark &
        \xmark \\
        \hline

        \multicolumn{2}{|l||}{CareerBuilder} &
        \xmark &
        \xmark &
        \cmark &
        \xmark \\
        \hline

        \multicolumn{2}{|l||}{ConfitV2} &
        \cmark &
        \cmark &
        \cmark &
        \xmark \\

        \hline
        \multicolumn{2}{|l||}{Approach of this thesis} &
        \cmark &
        \cmark &
        \xmark &
        \cmark \\

        \hline

    \end{tabular*}
\end{table}


In summary, although existing systems achieve strong performance in specific settings, there remains no approach that is semantically robust, deployable using lightweight, fully open‑source components, and architecturally transparent. This thesis addresses this gap by proposing a job‑matching system that employs small open‑source LLMs for structured requirements extraction together with embedding‑based similarity search, thereby offering an accessible, computationally efficient, and interpretable-by-design alternative.

\chapter{Conception}
Building on the identified research gap for an accessible yet powerful job matching framework, this chapter details the conception of the proposed filtering and matching system. It addresses the challenges and objectives outlined in the introduction. The system functions as a processing layer that builds upon the data acquisition capabilities of the YourJobFinder platform. The technical scope of this work begins after YourJobFinder has aggregated the raw, unstructured job descriptions based on an initial user query.

The architecture of the proposed system is illustrated in Figure \ref{fig:yourjobfinder_architecture}. The process is initiated when the web crawler provides raw HTML text to the backend. The work presented in this thesis focuses on the subsequent stages, which include a multi-stage pipeline designed to preprocess the data, extract structured requirements using a LLM, and perform a similarity search to identify relevant job postings. This chapter describes the two primary components of this framework: the requirements extraction process and the embedding-driven similarity search.

\begin{figure}[h!]
    \centering
    \includegraphics[width=1\textwidth]{images/DDD-Miro-7-11-bold}
    \caption{Diagram of the proposed system architecture}
    \label{fig:yourjobfinder_architecture}
\end{figure}

\section{Handling raw data}
The data pipeline begins with raw HTML job postings, as shown in Figure \ref{fig:yourjobfinder_architecture}. To eliminate irrelevant data like scripts and style tags and create a clean input for the LLM, the raw HTML is converted to Markdown using the markdownify library \cite{python-markdownify}. This pre-processing step ensures that the model focuses solely on the semantic content of the job description for requirements extraction.


\section{Schema based LLM Output}
A significant challenge in utilizing LLMs for information extraction is the non-deterministic nature of their output. To overcome this inconsistency and ensure all extracted data is machine-readable and standardized, a schema-guided generation approach is adopted. The outlines library \cite{willard2023outlines} is integrated into the system to force the LLM to generate responses that strictly adhere to a predefined JSON schema, which will be discussed further in the next chapter.  This method guarantees structural consistency across all processed job descriptions, which is crucial for the subsequent similarity search stage.

\section{LLM Models}
The models selected for the benchmark evaluation represent state-of-the-art open-source models as of October 2025:

\begin{table}[h]
	\caption{Model Parameter Sizes}
	\label{tab:model_params}
	\centering
	\begin{tabular}{|l|c|}
		\hline
		\textbf{Model Name} & \textbf{Parameter Size} \\
		\hline
		\hline
		Qwen3 \cite{qwen3} & 8B  \\
		\hline
		GLM-4-0414 \cite{glm2024chatglm} & 9B \\
		\hline
		mistral-7B-instruct \cite{mistral-7b} & 7B  \\
		\hline
		Llama-3.1-8B-Instruct \cite{llama-3} & 8B  \\
		\hline
	\end{tabular}
\end{table}

\section{Managing Input Data}
This thesis employs a fixed-size chunking strategy. While semantic chunking represents a more sophisticated alternative designed to preserve contextual boundaries, recent research indicates that its significant computational overhead does not justify the potential performance gains \cite{semantic-chunking}. Therefore, fixed-size chunking was selected for its pragmatic balance of efficiency and effectiveness.

\subsection{Fixed-size chunking}
The number of tokens in a chunk and, optionally, whether there should be any overlap between them, is determined based on a fixed rule. In general, some overlap is desirable to prevent loss of semantic context between chunks. Fixed-size chunking is computationally efficient and simple to implement since it does not require any specialized techniques or libraries \cite{tamingllms}.


\section{LLM Output Evaluation}
A key challenge in evaluating the extracted requirements is the semantic variability inherent in LLM-generated text. A single requirement can be formulated in numerous, lexically distinct ways while remaining semantically correct. For instance, the requirement `Proven ability to manage projects simultaneously' could be correctly summarized as `Project management' or `Experience managing multiple projects'. Consequently, relying on a fixed, manually annotated ground truth becomes unreliable, since semantically equivalent requirements expressed with different wording would be incorrectly treated as mismatches.

To address this challenge, this thesis adopts an automated and dynamic evaluation strategy based on the 'LLM-as-a-judge' paradigm \cite{llm-as-a-judge}. Instead of relying on exact string matching, this approach uses a powerful LLM to assess the semantic faithfulness of the generated output against the source text. As Huyen mentions, ``AI judges are fast, easy to use, and relatively cheap compared to human evaluators. They can also work without reference data'' \cite{ai-engineering-book}. For this purpose, the DeepEval framework was employed \cite{deepeval}. DeepEval utilizes its G-Eval metric to prompt a judge model, which scores the quality of the extracted JSON object relative to the original job description \cite{G-eval}. The specific metrics guiding this evaluation will be discussed in the following section.

\subsection{LLM-as-a-Judge}
``The approach of using AI to evaluate AI is called AI as a judge or LLM as a judge'' \cite{ai-engineering-book}.
``An emerging method for evaluating generated content involves using other LLMs as evaluators'' \cite{blackbox-llmasjudge}. ``These evaluations can take various forms, including explanations, numeric values, or categorical rating'' \cite{blackbox-llmasjudge}. Using LLM-as-a-Judge is not just cost-effective, `` Studies have shown that certain AI judges are strongly correlated to human evaluators.'' \cite{ai-engineering-book}. Zheng et al. found in their work that LLM Judges and humans can achieve 85\% agreement, while the agreement among human evaluators was 81\% \cite{llm-as-a-judge}.


\subsection{G-Eval}
``G-EVAL is a prompt-based evaluator with three main components: 1) a prompt that contains the definition of the evaluation task and the desired evaluation criteria, 2) a chain-of-thoughts (CoT) that is a set of intermediate instructions generated by the LLM describing the detailed evaluation steps, and
3) a scoring function that calls LLM and calculates the score based on the probabilities of the return tokens.'' \cite{G-eval}

\subsection{Metrics} \label{metrics}
The LLM-as-a-judge evaluation was guided by four key metrics, adapted from the framework proposed by Dong Yuan et. al \cite{evaluation-mertric}. Each metric was specifically interpreted for the task of requirements extraction as follows:

\paragraph{Correctness} In the context of this thesis, this measure is used to evaluate whether the extracted job requirements are accurate and actually present in the input text.

\paragraph{Completeness} this metric is used to evaluate whether all mandatory job requirements from the input text were extracted.

\paragraph{Alignment} Applied to this task, alignment evaluates whether the model correctly avoided extracting `nice-to-have', `preferred', or `bonus' requirements.

\paragraph{Readability} Here, this metric is adapted to assess whether the extracted requirements are properly categorized as given into skills, experiences, and qualifications, and whether the output is well-structured, for instance, if it is free of syntax errors. This aspect will be explained deeper in the implementation section.


\section{Embedding and Similarity Search}
Subsequent to the extraction and structuring of job requirements,the system proceeds to the second stage, which addresses the objective of dynamic job matching. To accomplish this, the structured job data and user query are encoded into vector representations. The all-MiniLM-L6-v2 sentence transformer model was selected for this task due to its strong zero-shot retrieval performance in job matching as mentioned in \Ref{introduction:similaritysearch}. This process is designed to identify the top 10 most relevant job job postings based on semantic similarity, thereby providing a more accurate matching system than title-based filtering.

\chapter{Implementation}
\label{ch:implementation}

This chapter details the technical implementation of the concept presented in the previous chapter. It begins by describing the development environment. Following this, \Autoref{sec:impl-data-processing} explains the data preprocessing pipeline. \Autoref{sec:impl-lre} presents the implementation of LRE and the evaluation framework using LLM-as-a-judge. Finally, \Autoref{sec:impl-tss} details the implementation of TSS.

\section{Development Environment} \label{sec:impl-dev-env}
The system was implemented using the Python programming language (Version 3.10), chosen for its extensive ecosystem of data science and machine learning libraries. The core dependencies include:
\begin{itemize}
	\item \textbf{Markdownify}: For converting raw HTML content into structured Markdown text.
	\item \textbf{Outlines}: To enforce structured JSON / Pydantic Model generation from the LLMs, ensuring deterministic output formats.
	\item \textbf{FastAPI}: For serving the inference engine as a scalable web service.
	\item \textbf{Sentence-Transformers}: For generating vector embeddings of requirements and user profile.
	\item \textbf{DeepEval}: For the automated LLM-as-a-judge evaluation.
\end{itemize}
All experiments and inference tasks were conducted on a dual GPU setup (2x 12GB NVIDIA RTX 3080), ensuring consistent performance benchmarks across all tested models. \Autoref{fig:system_architecture} illustrates the high-level architecture of the implemented system.

\begin{figure}[h]
	\centering
	\includegraphics[width=\textwidth]{images/job-matching-system-architecture}
	\caption{System Architecture}
	\label{fig:system_architecture}
\end{figure}


\section{Data Preprocessing} \label{sec:impl-data-processing}
As outlined in the conception, the raw data consists of HTML documents. The preprocessing pipeline transforms this unstructured input into clean, chunked text suitable for LLM processing.

\subsection{HTML to Markdown Conversion}
The markdownify library is configured to strip unnecessary tags (such as scripts, styles, and images) while preserving the structural hierarchy of the document. Headings, lists, and paragraphs are retained, as these structural elements provide critical semantic cues for the extraction model. This reduction significantly lowers the token count while maintaining the semantic structure of the job posting.

\subsection{Hybrid Chunking Strategy} \label{subsec:impl-chunking}
To process job descriptions that exceed the context window of smaller LLMs, a hybrid chunking strategy is implemented. The text is primarily split based on Markdown headers (using the regex \verb|r'(#{1,6}\s+.*?\n)'|). This ensures that logical sections, such as Responsibilities or Requirements, remain intact within a single context window.

A fallback mechanism is implemented for cases where a section exceeds the maximum token limit. If a semantic chunk exceeds the configured size (defaulting to approximately 12,000 tokens, estimated via character count), it is further divided based on a strict character limit to prevent memory overflows during inference.


\section{Requirements Extraction} \label{sec:impl-lre}
The core component of the system is the extraction of structured data. To address the non-deterministic nature of LLMs, the outlines library is employed to constrain the generation process using FSM logic.

\subsection{Output Structure Definition}
A strict Pydantic model defines the output schema. This ensures that every processed job description yields a consistent data structure, regardless of the underlying LLM used. The schema is defined as follows:
\begin{center}
	\begin{lstlisting}[language=python, caption={Pydantic Schema for Requirements Extraction}, label={lst:requirements_schema}]

from typing import List
from pydantic import BaseModel, Field

class Requirements(BaseModel):
    """Normalized requirements schema for job extraction outputs."""
    skills: List[str] = Field(
        default_factory=list,
        description="List of required skills"
    )
    experiences: List[str] = Field(
        default_factory=list,
        description="List of experience requirements"
    )
    qualifications: List[str] = Field(
        default_factory=list,
        description="List of qualifications/certifications"
    )
    \end{lstlisting}
\end{center}

\subsection{Prompt Engineering}
The prompt provided to the LLM consists of a system instruction defining the persona and the task, followed by the specific job description chunk. For effective job matching, the system focuses exclusively on must-have requirements, as these are the essential prerequisites a candidate needs to apply for the job. \Autoref{fig:prompt} shows the system prompt, which requires the LLM to strictly adhere to this constraint, be concise, and only respond in the defined categories.

\begin{figure}[h]
	\centering
	\includegraphics[width=0.8\textwidth]{images/prompt}
	\caption{System prompt used for requirements extraction}
	\label{fig:prompt}
\end{figure}


\subsection{Inference}
The inference logic is encapsulated within a generic \texttt{LLMExtractor} class, designed to be model-agnostic. This allows for seamless switching between different open-source models by simply updating a configuration dictionary.

The extractor initializes the model and tokenizer and wraps them with the \texttt{Generator} class from the outlines library. Passing the \texttt{Requirements} object defined in \Autoref{lst:requirements_schema} instructs the \texttt{Generator} to produce only outputs that match that schema. The implementation of this logic is shown in \Autoref{code:api-implementation}.

\begin{center}
	\begin{lstlisting}[language=python, caption={LLM Extractor Implementation}, label={code:api-implementation},float=h]
class LLMExtractor:
    def __init__(self, model_id, chunk_size, device_kwargs=None):
        self.model_id = model_id
        self.chunk_size = chunk_size
        self.device_kwargs = device_kwargs or {}
        self.generator = None
        self._load_model()

    def _load_model(self):
        # ... Memory management and cleanup ...
        self.tokenizer = AutoTokenizer.from_pretrained(self.model_id)
        self.model = AutoModelForCausalLM.from_pretrained(
            self.model_id,
            **self.device_kwargs,
            trust_remote_code=True,
        )
        # Initialize Outlines generator with schema constraint
        outlines_model = from_transformers(self.model, self.tokenizer)
        self.generator = Generator(outlines_model, Requirements)
    \end{lstlisting}
\end{center}

This design simplifies the process of managing models. Adding a new model or removing an existing one requires modifying the JSON entries. For example, \Autoref{lst:llm-defintion} shows how to add the Qwen3-8B model.

\begin{center}
	\begin{lstlisting}[language=json, caption={LLM Definition}, label={lst:llm-defintion},float=h]
{
  "qwen3-8b": {
    "model_id": "Qwen/Qwen3-8B",
    "chunk_size": 12000,
    "device_kwargs": {
      "device_map": "auto",
      "dtype": "torch.bfloat16"
    }
  }
}
    \end{lstlisting}
\end{center}

\subsection{Exception handling} \label{subsec:exception_handling}

After the output is generated, the output structure is validated against the Pydantic model using the built-in \texttt{validate\_json()} method.
 While the outlines library is designed to ensure that the model output follows the schema shown in \Autoref{lst:requirements_schema}, issues can still occur during generation. For example, if the model reaches its \texttt{max\_new\_tokens} limit, it may produce incomplete or invalid JSON that does not pass validation.
When this happens, the system handles the exception and returns a \texttt{Requirements} object in which the fields are empty lists.
This approach allows the rest of the pipeline to run without failing, but it also means that there is no clear indication that the output was invalid, which makes troubleshooting more challenging.

\subsection{Result Aggregation}
Since job descriptions are processed in chunks, a single job posting may yield multiple \texttt{Requirements} objects. An aggregation step is implemented to merge these partial results. A \texttt{defaultdict} of sets is utilized to collect skills, experience, and qualifications across all chunks. The use of sets automatically handles deduplication, ensuring that if a skill is mentioned in multiple sections, it appears only once in the final output.
\subsection{G-Eval Metrics}
To integrate the evaluation strategy described in \Autoref{conception:g-eval}, this work implements four metrics defined in \Autoref{metrics}. Each metric specifies a natural-language description of the goal, a sequence of evaluation steps, the relevant test case fields (job description and/or extracted requirements), and a fixed success threshold of 0.7. \Autoref{code:geval-metrics} shows an example metric definition.

\begin{center}
	\begin{lstlisting}[language=python, caption={Implementation of a G-Eval metric}, label={code:geval-metrics}]
from deepeval.metrics import GEval

def create_evaluation_metrics(model: DeepEvalBaseLLM) -> List[GEval]:
    correctness_metric = GEval(
        name="Correctness",
        criteria=(
        "Evaluate whether the extracted job requirements are"
        "accurate and actually present in the input text."
        ),
        evaluation_steps=[
            "Read the input job description carefully",
            "Compare each extracted requirement with the input text",
            # further evaluation steps ...
        ],
        evaluation_params=[
            LLMTestCaseParams.INPUT,
            LLMTestCaseParams.ACTUAL_OUTPUT,
        ],
        model=model,
        threshold=0.7,
    )
    # additional metrics: Completeness, Alignment , Readability
    return [correctness_metric, ...]
    \end{lstlisting}
\end{center}

The evaluation pipeline first loads all JSON outputs from the requirements extraction stage and converts them into \texttt{LLMTestCase} objects, where the original job description serves as the input and the serialized \texttt{Requirements} object is used as \texttt{actual\_output}. Finally, the system utilizes DeepEval's \texttt{evaluate} function and a judge model with external api to execute these metrics asynchronously over all test cases, collecting the resulting scores.

\section{Embedding and Similarity Search} \label{sec:impl-tss}
This section describes the implementation of the similarity matching component introduced in the conception. The goal is to compute a semantic similarity score between extracted job requirements and a user profile.

\subsection{Vector Embedding}
The embedding functionality is implemented using the \texttt{sentence-transformers} library, which provides pre-trained transformer models optimized for semantic similarity tasks \cite{sentence-bert}. The model all-MiniLM-L6-v2 is loaded once as a singleton to avoid repeated initialization overhead during inference.

\subsection{MaxSim}
The MaxSim algorithm described in \Autoref{subsec:maxsim} is implemented in the \texttt{compute\_maxsim} function. The implementation directly follows the conceptual formulation: for each job requirement, find the maximum similarity to any user profile item, then average these maximum scores.
\begin{center}
    \begin{lstlisting}[language=python, caption={Implementation of MaxSim Logic}, label={code:compute_maxsim}]
def compute_maxsim(user_items: List[str],
                   job_items: List[str],
                   model: SentenceTransformer) -> float:
    if not user_items or not job_items:
        return 0.0

    # 1. Encoding
    user_embeddings = model.encode(user_items, convert_to_tensor=True)
    job_embeddings = model.encode(job_items, convert_to_tensor=True)

    # 2. Similarity Matrix
    cosine_scores = util.cos_sim(job_embeddings, user_embeddings)

    # 3. Max-Over-Rows (dim=1)
    max_scores, _ = torch.max(cosine_scores, dim=1)

    # 4. Aggregation
    return max_scores.mean().item()
    \end{lstlisting}
\end{center}
The \texttt{util.cos\_sim} function from \texttt{sentence-transformers} computes the full similarity matrix $C$ between all job and user embeddings. The \texttt{torch.max} operation along dimension 1 extracts the maximum similarity for each job requirement (row), and \texttt{mean()} computes the final field score as defined in \Autoref{eq:maxsim_coverage}.

\subsection{Weighted Aggregation}
\Autoref{lst:weighted-agg} shows how final similarity score is combined to the MaxSim scores from all three fields using configurable weights. By default, skills receive a weight of 0.5, experiences 0.3, and qualifications 0.2, reflecting the prioritization of skills-based matching discussed in \Autoref{sub:conception-weighted-field}.

\begin{center}
	\begin{lstlisting}[language=python, caption={Weighted score aggregation}, label={lst:weighted-agg}]
def compute_similarity(user_profile, extracted_requirements,
                       weights=None) -> SimilarityScore:
    if weights is None:
        weights = {"skills": 0.5, "experiences": 0.3,
                   "qualifications": 0.2}

    model = get_model()
    overall_scores = {}

    for field in ['skills', 'experiences', 'qualifications']:
        user_items = getattr(user_profile, field, [])
        job_items = getattr(extracted_requirements, field, [])
        overall_scores[field] = compute_maxsim(user_items,
                                                job_items, model)

    weighted_score = sum(
        overall_scores[field] * weights.get(field, 0.0)
        for field in overall_scores
    )
    return SimilarityScore(score=weighted_score)
    \end{lstlisting}
\end{center}

This design allows the weights to be adjusted via the API without modifying the codebase, enabling experimentation with different prioritization strategies.

\chapter{Evaluation}
This chapter presents a quantitative evaluation of the selected open-source Large Language Models to determine their effectiveness in extracting structured job requirements from unstructured text. This analysis directly addresses the second research question (RQ2) by assessing which model demonstrates the best performance in terms of correctness, completeness, readability, and alignment. The performance of each model is measured across three dimensions: (i) an overall model ranking derived from aggregated scores, (ii) a detailed comparison of average scores for each of the four evaluation metrics, and (iii) an analysis of task completion pass rates. The presented results provide the foundation for the subsequent summary of findings and the discussion of future research directions in the Outlook.


\section{Model Performance Comparison}
In this section, we evaluate the performance of the different models based on three key metrics: overall model ranking, passed rates, and average scores. The following figures illustrate the comparative performance of the models in each of these areas.

\begin{figure}[h]
    \centering
    \includegraphics[width=0.8\textwidth]{images/overall_model_ranking}
    \caption{Overall model ranking}
    \label{fig:overall_model_ranking}
\end{figure}

Figure \ref{fig:overall_model_ranking} presents the overall ranking of the models. This ranking is an overall score derived from multiple metrics as discussed before see \Ref{metrics}. As can be seen, Qwen3-8B achieves the highest rank, followed by GLM-4. This suggests that, in aggregate, Qwen3-8B demonstrates superior performance across the evaluated criteria.


\begin{figure}[h]
    \centering
    \includegraphics[width=0.8\textwidth]{images/average_scores_comparison}
    \caption{Overall average performance score of the evaluated LLM models across all four metrics}
    \label{fig:average_scores_comparison}
\end{figure}

Figure \ref{fig:average_scores_comparison} shows comparison of  the average scores obtained by each model in all metrics. The score is a measure of the quality of the model's output, with higher scores indicating better performance. The results in this figure are consistent with the previous metrics, showing Model Qwen3-8B with the highest average score. This further supports the conclusion that Model Qwen3-8B is the most effective model in this evaluation. It is important to note that Qwen3-8B has better percentage in Alignment category than GLM-4 but lower percentage in Completeness, which means that Qwen3-8B identifies 'must-have' requirements better but hallucinates more compared to GLM-4.

\begin{figure}[h]
    \centering
    \includegraphics[width=0.8\textwidth]{images/passed_rates_comparison}
    \caption{Passed rates comparison}
    \label{fig:passed_rates_comparison}
\end{figure}

Figure \ref{fig:passed_rates_comparison} shows a comparison of the passed rates for each model in all metrics. G-Eval has the option to define a success threshold, which was set to 0.7 for all metrics. Based on that the passed rate is defined as the percentage of test cases that the model successfully completes. The chart indicates that Model Qwen3-8B has the highest passed rate, reinforcing its strong performance. Models GLM-4 and mistral-7B-instruct have nearly the same passed rate



\section{Summary of Evaluation}

\section{Outlook}

\chapter{Conclusion}

\begin{enumerate}
    \item \textbf{Architecture over Size}: Despite being the oldest and smallest model in the benchmark, Mistral-7B (Overall Score: 0.46) outperformed the newer Llama-3.1-8B. This indicates that for specific information extraction tasks, model architecture and training data quality may be more decisive than raw parameter count or release date.

    \item \textbf{Readiness for Automation}: While Qwen3-8B achieved the top rank, the absolute scores (hovering around 0.5-0.6) and low pass rates (<30\% for strict correctness) indicate that 8B-parameter models are not yet fully reliable for fully autonomous, unsupervised extraction. They require either human-in-the-loop verification or further fine-tuning to reach production-grade reliability.
\end{enumerate}

\section{Outlook}
\begin{enumerate}
\item Thinking models are not considered, Data Processing was not perfect (data imbalance),
\item bigger models,
\item  Readability metric for evaluation was not so effective, because the for exception handling it retuned an empty object, so there were no output that had syntax errors.
    \item
    \item How fast the whole process is has not been measured.
\end{enumerate}


\cleardoublepage

% VERZEICHNISSE (Abbildungen, Tabellen)
% Literatur 
\bibliographystyle{alphadin}
\bibliography{Ba}
\cleardoublepage

% ERKLÄRUNG
% \chapter*{Eidesstattliche Versicherung}
\thispagestyle{empty}
\addcontentsline{toc}{chapter}{Eidesstattliche Versicherung}

Hiermit versichere ich an Eides statt, dass ich die vorliegende Arbeit im Bachelorstudiengang Wirtschaftsinformatik selbstständig verfasst und keine anderen als die angegebenen Hilfsmittel – insbesondere keine im Quellenverzeichnis nicht benannten Internet-Quellen – benutzt habe. Alle Stellen, die wörtlich oder sinngemäß aus Veröffentlichungen entnommen wurden, sind als solche kenntlich gemacht. Ich versichere weiterhin, dass ich die Arbeit vorher nicht in einem anderen Prüfungsverfahren eingereicht habe. Sofern im Zuge der Erstellung der vorliegenden Abschlussarbeit generative Künstliche Intelligenz (gKI) basierte elektronische Hilfsmittel verwendet wurden, versichere ich, dass meine eigene Leistung im Vordergrund stand und dass eine vollständige Dokumentation aller verwendeten Hilfsmittel gemäß der Guten Wissenschaftlichen Praxis vorliegt. Ich trage die Verantwortung für eventuell durch die gKI generierte fehlerhafte oder verzerrte Inhalte, fehlerhafte Referenzen, Verstöße gegen das Datenschutz- und Urheberrecht oder Plagiate.

\noindent Ich stimme der Einstellung der Arbeit in die Bibliothek des Fachbereichs Informatik zu.

\vspace{2cm} 

\noindent Hamburg, den \uline{~~~~~~~~~~~~~~~~~~~~}~~~~~Unterschrift: \uline{~~~~~~~~~~~~~~~~~~~~~~~~~~~~~~~~~~~~~~~~~~~~~~~~~~} 

    
\end{document}