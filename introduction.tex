\chapter{Introduction} \label{ch:introduction}
In the digital age, online job platforms provide users with access to a wide range of employment opportunities. Yet, despite this abundance, finding positions that genuinely match a user's skills and preferences remains a challenge \cite{challenge_recruitment}. A central cause of this mismatch lies in the fundamental search and filtering mechanisms employed by many platforms, which rely heavily on job titles as the primary means of retrieval \cite{challenge_recruitment}.
Title-based search suffers from a well‑documented limitation: job titles alone rarely capture the full semantic content of a role. As a result, users frequently encounter postings that are nominally relevant but misaligned with the actual tasks and qualifications required \cite{kabir2025ordinal, womark2025lessons}. In addition, ranking systems on some platforms are influenced by mechanisms such as bid dominance, where less relevant jobs appear prominently simply because employers submit higher bids \cite{tradingrelevancerevenuejobs}. This not only distorts search results but also reduces the overall quality and trustworthiness of the user experience.

These issues demonstrate that existing job‑matching systems are often
insufficiently effective and motivate the need for more robust,
semantically informed approaches—an issue at the core of this thesis.



\section{Motivation}
TODO: explain that data is spreaded over multiple platforms and yourjobfinder solves it by scraping multiple platforms but still a problem remains which is that it gets results which are not relevant because 1. title based search is not complete (covers all aspects) 2. bid dominance \\
As industries evolve, job roles and the competencies they require have
become increasingly specialized. Musazade et al. observe that ``data professionals have started to be required to possess a narrower competence and specialization in particular common, unique and extensive tools and technologies'' \cite{Musazade2023toolsAndTech}. This heightened specialization magnifies the weaknesses of title‑based search systems: even slight variations in wording can hide otherwise highly similar job roles from search results.
Kabir et al. illustrate this issue with queries such as “software engineer Java” and “Java backend developer,” which differ lexically but are semantically aligned due to overlapping skill requirements \cite{kabir2025ordinal}. Similarly, Womark et al. highlight cases where generic queries like “Human Resources” lead to irrelevant results, such as job descriptions containing phrases like “speak to a Human Resources representative,” without actually referring to HR roles \cite{womark2025lessons}.
To address these limitations, researchers increasingly advocate for the use of contextual information. Kabir et al., for instance, recommend incorporating job‑related skills, industries, and other content-rich signals to improve similarity prediction models \cite{kabir2025ordinal}. Building upon this insight, the central idea of this thesis is to analyze full job descriptions and extract structured information to enable semantically grounded matching.

This work builds on YourJobFinder \cite{yourjobfinder}, an existing aggregation tool that scrapes unstructured job descriptions from multiple online sources. While useful for data collection, its reliance on title-based retrieval results in low precision and many irrelevant postings. To address this, the core contribution of this thesis is a filtering and matching system that processes these unstructured descriptions and identifies postings that genuinely align with a user’s profile.

The system developed in this thesis uses an open-source Large Language Model (LLM) to extract key attributes—such as skills, experience, and qualifications—from job descriptions. These extracted requirements are then transformed into vector representations and compared with similarly vectorized user profiles. By relying solely on open-source models and tools, the approach aims to provide a transparent, scalable, and cost‑effective solution.

\section{Research Questions}

To guide this research and address the stated objectives, the thesis investigates the following questions:

\begin{itemize}
    \item\label{itm:rq1}RQ1: How effectively can open-source Large Language Models extract structured job requirements (skills, experience, qualifications) from unstructured job descriptions?

    \item\label{itm:rq2} RQ2: How does the approach proposed in this thesis compare to conventional job search engines with respect to the relevance and accuracy of retrieved job postings?
\end{itemize}

