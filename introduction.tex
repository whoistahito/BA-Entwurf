\chapter{Introduction}
In the digital age, the proliferation of online job platforms has created a paradox: while job seekers have access to an unprecedented volume of opportunities, finding truly suitable positions remains a significant challenge \cite{challenge_recruitment}. One of the issues stems from the  fundamental search and filtering mechanisms employed by many platforms \cite{challenge_recruitment}. The predominant use of job titles as a search filter frequently leads to a mismatch between the search query and the functional requirements of the resulting job listings, which occurs because titles alone cannot capture the semantic similarity between a user's query and the actual tasks described in a role \cite{kabir2025ordinal}, presenting users with  job postings that are nominally correct but a poor match for their actual skills and qualifications \cite{womark2025lessons}. Another key problem is a phenomenon known as bid dominance, where less relevant jobs are ranked higher due to their higher submitted bids \cite{tradingrelevancerevenuejobs}.

\section{Motivation}
As technology evolves, jobs and their corresponding requirements become increasingly specific. Musazade et al. highlight this trend, noting that ``the study suggests that data professionals have started to be required to possess a narrower competence and specialization in particular common, unique and extensive tools and technologies'' \cite{Musazade2023toolsAndTech}. As pointed out previously, this increasing specialization creates challenges for online job platforms, which often struggle with matching jobs to users who fit their profiles. For example, Kabir et al. mention in their work that ``two queries such as `software engineer Java' and `Java backend developer' might have different token representations but are semantically similar due to overlapping skill requirements'' \cite{kabir2025ordinal}. Another example is from Womark et al., who note that ``the query `Human Resources' often returned unrelated jobs matching the term only in descriptions, in clauses such as `speak to a Human Resources Rep'.'' \cite{womark2025lessons}.\\
To overcome the limitations of title-based search, Kabir et al. suggest that ``contextual information, which provides a richer context to compare two job queries, such as skills associated with a job, or the specific industry sector to which the job belongs to, need to be incorporated in the similarity prediction model'' \cite{kabir2025ordinal}. Based on this, the approach of this thesis is to analyze the full text of job descriptions and match them based on a similarity search. \\

This research builds upon YourJobFinder, a pre-existing job aggregation tool, for the initial data acquisition phase. YourJobFinder scrapes unstructured job descriptions from various online sources based on a job title. However, this initial collection, which relies solely on title-based matching, suffers from low precision and includes many irrelevant postings. The core technical contribution of this thesis is a novel filtering and matching system that addresses this limitation. This system takes the raw, unstructured text provided by YourJobFinder as input and applies an intelligent filtering to identify job postings that match a user's profile.\\

Therefore, the objective of this thesis is to design and implement a job matching system that uses a Large Language Model (LLM) to extract key attributes—such as skills, experience, and qualifications—from job descriptions. This information is then converted into a vector representation. The system performs a comparison between the job's vector and a similarly vectorized user profile. This thesis aims to design an efficient system capable of producing accurate and reliable results using open-source models and tools, ensuring accessibility and cost-effectiveness.


\section{Questions}
To guide this research and systematically address the stated objective, this thesis poses the following questions:

\begin{itemize}
\item RQ1: How effectively can open-source Large Language Models extract structured job requirements (skills, experiences, qualifications) from unstructured job descriptions?


\item RQ2: Which open-source LLM demonstrates the best performance in terms of correctness, completeness, and alignment for the task of requirements extraction?


\end{itemize}

