\chapter{Introduction}
In the digital age, the proliferation of online job platforms has created a paradox: while job seekers have access to an unprecedented volume of opportunities, finding truly suitable positions remains a significant challenge \cite{challenge_recruitment}. One of the issues stems from the  fundamental search and filtering mechanisms employed by many platforms \cite{challenge_recruitment}. The predominant use of job titles as a search filter frequently leads to a mismatch between the search query and the functional requirements of the resulting job listings, which occurs because titles alone cannot capture the semantic similarity between a user's query and the actual tasks described in a role \cite{kabir2025ordinal}, presenting users with  job postings that are nominally correct but a poor match for their actual skills and qualifications \cite{womark2025lessons}. Another key problem is a phenomenon known as bid dominance, which is well explained by Pourbabaee et. al, where this phenomenon occurs when less relevant jobs are ranked higher
 due to their higher submitted bids \cite{tradingrelevancerevenuejobs}.

\section{Motivation}
As technology evolves, jobs and their corresponding requirements become increasingly specific. Musazade et al. highlight this trend, noting that ``the study suggests that data professionals have started to be required to possess a narrower competence and specialization in particular common, unique and extensive tools and technologies'' \cite{Musazade2023toolsAndTech}. As pointed out previously, this increasing specialization creates challenges for online job platforms, which often struggle with matching jobs to users who fit their profiles. For example, Kabir et al. mention in their work that ``two queries such as `software engineer Java' and `Java backend developer' might have different token representations but are semantically similar due to overlapping skill requirements'' \cite{kabir2025ordinal}. Another example is from Womark et al., who note that ``the query `Human Resources' often returned unrelated jobs matching the term only in descriptions, in clauses such as `speak to a Human Resources Rep'.'' \cite{womark2025lessons}.\\
To overcome the limitations of title-based search, Kabir et al. suggest that ``contextual information, which provides a richer context to compare two job queries, such as, skills associated to a job, or the specific industry sector to which the job belongs to, need to be incorporated in the similarity prediction model'' \cite{kabir2025ordinal} Based on this, the approach of this thesis is to analyze the full text of job descriptions and match them based on a similarity search. \\
To collect job descriptions, a foundational tool is needed. For this purpose, this thesis builds upon YourJobFinder, a pre-existing job aggregation tool that performs the initial scraping of job postings. The technical contribution of this work begins where YourJobFinder's role ends, focusing on the intelligent filtering and matching of the scraped, unstructured text. For any given user query, YourJobFinder performs the initial step of aggregating relevant job postings from various platforms and scraping their descriptions. This raw, unstructured text, gathered in real-time, then serves as the direct input for the core technical contribution of this thesis: a novel filtering and matching system. \\
While the YourJobFinder platform is capable of performing this initial aggregation, the resulting collection of job postings, based only on titles, is inherently lacking in precision. Simply gathering a large volume of listings does not solve the core challenge of matching a job posting with specific requirements to a user with a profile that contains skills, experiences and qualifications. For this important part, an intelligent filtering mechanism is needed.The objective of this thesis is to design and implement a job matching system that extracts necessary skills, experiences, and qualifications using a Large Language Model (LLM), converts that information into a vector, and then performs a comparison between this vector-represented information and a user profile, which has the same fields and is also represented in a vector space. This Thesis aims to design an efficient system capable of producing accurate and reliable results using open-source models and tools, ensuring accessibility and cost-effectiveness.

\section{Questions}

\begin{itemize}
\item RQ1: How effectively can open-source Large Language Models extract structured job requirements (skills, experiences, qualifications) from unstructured job descriptions?


\item RQ2: Which open-source LLM (among Qwen3-8B, GLM4-9B, etc.) demonstrates the best performance in terms of correctness, completeness, and alignment for the task of requirements extraction?


\end{itemize}

